\chapter*{Introduction} 
\addcontentsline{toc}{chapter}{Introduction}

The proposed solution to the previously mentioned problems has the name \textbf{Think-In}, and is a web application where virtual conferences or events can be held with the purpose of networking.
Having mentioned the two below issues, the \textbf{general purpose} of the application is to help people connect and network, therefore the users will expand their professional relationships and find new opportunities:

\begin{itemize}
    \item 
An issue that can appear in conferences of bigger sizes because of the diversity of the attenders, which other existent applications don't solve, is the \textbf{language barrier}. \textbf{Think-In} solves this issue as messages sent across the conference are translated in real-time in the user's native language or any other language the user desires. This feature can bring people together from all around the world as there are no communication bounds.
    \item
An inherent problem to real world conferences or events is that you don't know many of the people who attend, therefore the \textbf{Think-In} users have the option to provide more information about themselves such as an avatar, their job title and links to their social media accounts so they can be easily discovered and engaged with.
\end{itemize}

As a response to the disadvantages stated at the end of the previous chapter, \textbf{Think-In} is platform agnostic so the user can use the application no matter what device it is accessed from, it was designed to be blazing fast and very easy to use. In contrast to the aforementioned desktop applications, it is a web application, therefore the developers don't have to take care of how updates are brought to the users because everytime the user reloads the page the newest version of the application is used.

From a technical point of view, the application's \textbf{frontend} is built using the Next.js\footnote{No configuration React framework: \href{https://nextjs.org/}{https://nextjs.org/}} React\footnote{JavaScript library for building user interfaces: \href{https://reactjs.org/}{https://reactjs.org/}} framework and has a JAMstack\footnote{\href{https://jamstack.org/}{https://jamstack.org/}} architecture, the pages are built at deploy time and they are stored in a S3\footnote{AWS' object storage service: \href{https://aws.amazon.com/s3/}{https://aws.amazon.com/s3/}} bucket as a static website, from there on they are served from the CloudFront\footnote{AWS' CDN service: \href{https://aws.amazon.com/cloudfront/}{https://aws.amazon.com/cloudfront/}} CDN\footnote{Content Distribution Network}. This setup ensures a very fast TTFB\footnote{Time To First Byte} since there are no round trips to the server in order to render the page and the files required for the web page are fetched from one of the CDN's proxy servers, which, as a consequence of being close to the user, ensure fast file retrieval. The \textbf{backend} relies on a Socket.IO\footnote{Real-time communication library which uses WebSockets or HTTP as a fallback: \href{https://socket.io/}{https://socket.io/}} server ensuring the fastest communication possible when stage interactions happen or when messages are sent, as well as an AWS API Gateway\footnote{API management tool \href{https://aws.amazon.com/api-gateway/}{https://aws.amazon.com/api-gateway/}} that centralizes access to Lambda\footnote{AWS' event-driven serverless compute service: \href{https://aws.amazon.com/lambda/}{https://aws.amazon.com/lambda/}} functions, which act as a REST API when they are triggered by HTTP events. TypeScript\footnote{JavaScript superset language that adds static typing: \href{https://www.typescriptlang.org/}{https://www.typescriptlang.org/}} was the language of choice all around the application stack, this was motivated by the fact that the frontend had to be written in JavaScript and the opportunity to reuse code in the backend was hard not to take. Additionally, TypeScript brings static typing to JavaScript, that has proved to be extremely helpful in eliminating type errors as well as simplifying the developing experience thanks to type hints povided by modern IDEs. In order to simplify the deployment of the application, DevOps practices such as CI/CD\footnote{Continuous Integration and Continuous Deployment} and IaC\footnote{Infrastructure as Code} have been adopted.

%TODO: metodologia folosită, structura lucrării (titlul capitolelor și legătura dintre ele).